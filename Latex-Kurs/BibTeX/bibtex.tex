\documentclass{scrartcl}
\usepackage[ngerman]{babel} 
\usepackage[T1]{fontenc} 
\usepackage[utf8]{inputenc}
\usepackage[style=authoryear, sorting=nty, maxbibnames=20, maxcitenames=3, backref]{biblatex} 
\usepackage{csquotes}
\usepackage{hyperref}
\addbibresource{Knuth.bib}
\addbibresource{literatur.bib}
\begin{document}
\noindent Viel zu lesen über \TeX{} findet 
man im \TeX{}book \autocite{Knuth1984}.

\noindent 
Die Theorie des Auger-Effekts ist beschrieben bei \textcite{AAberg1982}
Zwei \glqq{}online\grqq{}-Einträge: \autocite{Kramida2015, Bezanson2015}
Ein Artikel mit vielen Autoren: \autocite{Uiberacker2007}

\printbibliography




\end{document}