\documentclass[a4paper, 10pt,onecolumn, captions=tableheading]{scrartcl}
\usepackage[ngerman]{babel}
\usepackage[T1]{fontenc}
\usepackage[utf8]{inputenc}
\usepackage{multirow}
%\usepackage{natbib}
\usepackage{graphicx}
\usepackage{amsmath, amssymb}
\usepackage{grffile} %einfacheres einbinden von Dateipfaden
\usepackage[hidelinks]{hyperref}
\usepackage{booktabs}
\setkomafont{sectioning}{\rmfamily}
\addtokomafont{author}{\scshape}

\title{Schlechte Bilder\thanks{Danke an John Wigg}} 
\author{Ich} %auch nach \begindocument möglich
\date{\today}
\setlength{\parindent}{0pt}

\begin{document}
\tableofcontents
\listoffigures
\listoftables

\section{Autokorrelation }
\subsection{Definition}

Die quadratische interferometrische Autokorrelationsfunktion $S_{quadAC}(\tau)$, definiert als
\begin{center}
	\begin{equation}
		S_{quadAC}(\tau)=\int_{-\infty}^{\infty}[E(t)+E(t-\tau)]^4\mathrm{d}t
		\label{eq:formel}
	\end{equation}
\end{center}
kann z.\,B. genutzt werden, um kurze Pulse zu analysieren. Wie man in Gleichung \eqref{eq:formel} sehen kann muss man für die Autokorrelation von $-\infty$ bis $\infty$ integrieren.

\subsection{Messung}
Tabelle \ref{Tabelle1} zeigt Messungen der Autokorrelation.
\begin{table}[h!]
\centering
\caption{Autokorrelation zu den Zeitpunkten $\tau =0$  und $\tau=\tau_1$}
\label{Tabelle1}
\begin{tabular}{|c|c|}
\hline \hline
$S_{quadAC}(0)$ & $S_{quadAC}(\tau_1)$\\ \hline
$0.001 V^4sm^{-4}$ & $0.32 V^4sm^{-4}$\\ \hline \hline
\end{tabular}
\end{table}
\section{Ionenerträge}

In \ref{Tabelle2} (auf Seite 2) sind Ionenerträge aufgeführt. Die Tabelle ist dem Artikel \cite{Stock} entnommen und leicht verändert.

\begin{table}[!h]
\centering
\caption{Relative Ionenerträge nach resonanter 1$s^{-1}3p-$ Anregung in Neon. Neben den berechneten Werten sind die experimentellen Daten von Morgan \textit{et al.} \cite{Morgan} aufgeführt.}
\label{Tabelle2}
\begin{tabular}{ccc}
\hline \hline
&\multicolumn{2}{c}{Ionenerträge}\\ \cline{2-3}
Ion & Theorie & Exp. [2] \\ \hline
N$e^1+$ & 0.74 & 0.65 $\pm$ 0.02\\
N$e^2+$ & 0.26 & 0.31 $\pm$ 0.02 \\ 
N$e^3+$ & -- & 0.03 $\pm$ 0.01\\
N$e^4+$ & -- & 0.002\\
\hline \hline

\end{tabular}
\end{table}

Dagegen mit Booktabs: 

\begin{center}
%booktabs
\begin{tabular}{lcc} \toprule & \multicolumn{2}{c}{Ionenerträge} \\ \cmidrule{2-3} Ion & Theorie & Exp.~\cite{Mor97} \\ \midrule Ne$^{1+}$ & $0.74$ & $0.65\pm0.02$ \\ Ne$^{2+}$ & $0.26$ & $0.31\pm0.02$ \\ Ne$^{3+}$ & --- & $0.03\pm0.01$ \\ Ne$^{4+}$ & --- & $0.002$ \\ \bottomrule \end{tabular}
\end{center}


\begin{figure}
\centering
\includegraphics[scale=0.25]{tauti.jpg}
\caption{Nicht die Sombrero Galaxie (M 104)}
\end{figure}

\section{Nicht die Sombrero-Galaxie}

Dieser Abschnitt hat nichts zu tun mit den Tabellen \ref{Tabelle1} und \ref{Tabelle2} oder der Gleichung \eqref{eq:formel}. Er handelt nicht von der Sombrero-Galaxie, Objekt Nummer M 104 im Messier-Katalog. Ein Bild der Galaxie kann in Abb. \ref{Tabelle1} auf Seite 2 bestaunt werden.
\newpage

\begin{thebibliography}{2}
	\bibitem{Stock} S.Stock, R. Beerwerth und S.Fritzsche: 
		\emph{Physical Review A} \textbf{95}, 053407 (2017)
	\bibitem{Morgan} D.V.Morgan, M.Sagurton und R.J. Bartlett: \emph{Physical Review A} \textbf{55}, 1113 (1997)
\end{thebibliography}


\end{document}
