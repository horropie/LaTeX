\documentclass[aspectratio=32]{beamer}
\usepackage[ngerman]{babel} 
\usepackage[T1]{fontenc} 
\usepackage[utf8]{inputenc}
\usepackage{lmodern}
\beamertemplatenavigationsymbolsempty
\usetheme{Bergen}
\usecolortheme{beaver}
\title{Meine erste \texttt{beamer}-Präsentation} 
\author{Ich} 
\institute{Friedrich-Schiller-Universität Jena} 
\date{\today}
\begin{document}
\begin{frame} \titlepage \end{frame}


\begin{frame}{Donald E. Knuth}{Biografie}
	\textbf{Donald Ervin \glqq Don\grqq\ Knuth} ist ein US-amerikanischer Informatiker. Er ist emeritierter \textit{Professor of the Art of Computer Programming} an der Stanford University.
	
\begin{itemize}
	\item geborgen am 10. Januar 1938 in Milwaukee, Wisconsin
	\item Master-Abschluss (1960) von der \textit{Case Western Reserve University}
	\item Doktortitel (1963) vom \textit{California Institute of Technology} 
	\item seit 1968 Professor für Informatik an der \textit{Stanford University}
	\item seit 1993 emeritiert
	
\end{itemize}
	
	
\end{frame}

\begin{frame}{Donald E. Knuth}{Werke}
	\begin{itemize}
		\item Donald Knuth ist Autor des Standardwerkes \textit{The Art of Computer Programming} und Urvater des Textsatzsystem \TeX 
		\item Mit seinem Buch \textit{Surreal Numbers: How Two Ex-Students Turned on to Pure Mathematics and Found Total Happiness} machte er die von John Horton Conway vorgestellten surrealen Zahlen populär.
		\item sein Multimediawerk \textit{Fastasia Apocalyptica} (Ein Orgelstück mit Videobegleitung) soll an seinem 80. Geburtstag in Pite\aa, Schweden uraufgeführt werden.
	\end{itemize}
\end{frame}

\begin{frame}{The Art of Computer Progamming}
	Donald Knuths \textit{magnum opus} ist die siebenteilige Reihe \textit{The Art of Computer Programming}, welche die Grundlagen der Computerprogrammierung behandelt:
	\begin{enumerate}
		\item Fundemental Algorithms (1968)
		\item Seminumerical Alglorithms (1969)
		\item Sorting and Searching (1973)
		\item Combinatorial Algorithms (2011)
		\item Sytactical Algorithms (geplant für 2025)
		\item The Theory of Context Free Languages
		\item Compilers
		
	\end{enumerate}
\end{frame}

\begin{frame}
	Diese Folie hat nichts mit Donald Knuth zu tun, sondern soll den Formelsatz demonstrieren.
	\par
	Mit der \textit{Multiconfigurations-Dirac-Fock-Methode} können Wellenfunktionen für Atome und Ionen generiert werden. Sie eignet sich insbesondere auch für Systeme mit mehreren offenen Schalen. \par
	
	Die Wellenfunktion $\psi_{\alpha}$ für ein Energielevel $\alpha$ wird dabei als Linearkombination sogenannter \textit{configuration state functions} $\Phi$ mit Parität $P$, Gesamtdrehimpuls $J$ und Projektion des Gesamtdrehimpulses $M$ konstruiert:
	\[
	\psi_{\alpha}(PJM)=\sum_{i=1}^{n_{c}}c_i(\alpha)\Phi(\gamma_iPJM)
	\]
\end{frame}

\begin{frame}{Die Sombrero-Galaxie}
	\begin{figure}
	\centering
	\caption{Die Sombrero-Galaxie (M104)}
		\include{lotze.jpg}
	\end{figure}
\end{frame}
\end{document}