\documentclass[a4paper, 10pt,onecolumn]{scrartcl}
\usepackage[ngerman]{babel}
\usepackage[T1]{fontenc}
\usepackage[utf8]{inputenc}

\title{Mein erstes \LaTeX-Dokument\thanks{John Wigg}} 
\author{Ich} %auch nach \begindocument möglich
\date{\today}
\setlength{\parindent}{0pt}

\begin{document}\tableofcontents
\maketitle
\section{Der erste Abschnitt}
\subsection{Der erste Unterabschnitt}

So fängt es also an, mein erstes \LaTeX-Dokument!\\
Sonderlich viel muss für den Anfang nicht drin stehen.

\paragraph{Mein erster Paragraph}

Hier gehört \textbf{\underline{\emph{Text}}} hin.
\subparagraph{Eine Aufzählung}
\begin{itemize}
	\item Ein erster Punkt
	\item \dots
	\item Ein letzter Punkt
	\begin{itemize}
		\item Ein Unterpunkt \footnote[10]{Eine Fußnote}
		\item Apostrophe \footnote{Als Code, sonst auch anders möglich} 
		\item \glqq{}--\grqq{} Mit Leerzeichen
		\item \glq\-\grq\, z\,B mit Abb.~1 gibt es auch geschützte Leerzeichen
		\item \glqq---\grqq Ohne Leerzeichen
	\end{itemize}
\end{itemize}


\end{document}