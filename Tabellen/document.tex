\documentclass[a4paper, 10pt,onecolumn]{scrartcl}
\usepackage[ngerman]{babel}
\usepackage[T1]{fontenc}
\usepackage[utf8]{inputenc}
\usepackage{multirow}
\usepackage{natbib}
\usepackage{graphicx}
\usepackage{amsmath, amssymb}


\title{Schlechte Tabellen\thanks{Danke an John Wigg}} 
\author{Ich} %auch nach \begindocument möglich
\date{\today}
\setlength{\parindent}{0pt}

\begin{document}
\tableofcontents
\maketitle

\section{Natur}
Die Natur beeindruckt durch ihre Vielfalt!\\

\begin{center}
	\begin{tabular}{|c|c|c|c|}
		\hline
		Bergketten & Flüsse & Seen & Strände und Wüsten\\
		\cline{1-3}
		\multirow{2}{*}{rau}   &  schnell & stehend & \multicolumn{1}{|r|}{sandig}       \\
		\cline{2-4}
		& Rehe & Vögel & Würmer\\
		\hline
	\end{tabular}
\end{center}

\section{Nährwertangaben}
Die Tabelle ist einer Packung Eiersalat-Brotaufstrich entnommen (leicht verändert):

\begin{center}
\begin{tabular}{|l|r|r|}
	\hline
	\textbf{Nährwerte} & \textbf{pro 100} & \textbf{(\%)}\footnotemark[1]\\
	\hline
	\multirow{2}{*}{Brennwert} & 1.176 kJ & \multirow{2}{*}{14\%} \\
	& 284 kcal & \\
	\hline 
	Fett & 25,7g & 37 \% \\
	davon gesättigte Fettsäuren & 3,4g & 17\% \\
	\hline
	Kohlenhydrate & 5,0g & 2\% \\
	davon Zucker & 3,0g & 3\% \\
	\hline
	Eiweiß & 7,8g & 16\% \\
	\hline
	Salz & 1,35g & 23\%\\ \hline
	Plutonium & \multicolumn{2}{c|}{hoffentlich keins}\\
	\hline

\end{tabular}
\end{center}
\footnotetext[1]{Referenzmenge für einen durchschnittlichen Erwachsenen (8.400 kJ/2.000 kcal)}
\newpage
Eine etwas ansprechendere Form der gleichen Tablle könnte z.\,B. so aussehen:

\begin{center}
	\begin{tabular}{l|r|r}
		\hline \hline
		\textbf{Nährwerte} & \textbf{pro 100} & \textbf{(\%)}\footnotemark[1]\\
		Brennwert & 1.176 kJ/284 kcal & 14 \%  \\
		Fett & 25,7g & 37 \% \\
		\quad davon gesättigte Fettsäuren & 3,4g & 17\% \\
		Kohlenhydrate & 5,0g & 2\% \\
		\quad davon Zucker & 3,0g & 3\% \\
		Eiweiß & 7,8g & 16\% \\
		Salz & 1,35g & 23\% \\
		Plutonium & \multicolumn{2}{c}{hoffentlich keins}\\
		\hline \hline
		
	\end{tabular}
\end{center}
\footnotetext[1]{Referenzmenge für einen durchschnittlichen Erwachsenen (8.400 kJ/2.000 kcal)}
\section{Matrizen in der Physik}

An vielen Stellen in der Physik benutzt man Matrizen. Die folgende Tabelle zeigt zwei Beispiele, die bei Koordinatentransformationen wichtig sind. Jede Spalte ist 6cm breit:

\begin{center}
	\begin{tabular}{|p{6cm}|p{6cm}|}
		\hline
		\multicolumn{1}{|c|}{Jacobi} & \multicolumn{1}{|c|}{Lorentz} \\
		\hline \hline
		Um das Flächenelement in Polarkoordinaten zu erhalten, benötigt man die entsprechende Jacobi-Determinante \textit{J}: 
		\begin{center}
		\[
		\mathrm{d}x\mathrm{d}y=J \mathrm{d}r\mathrm{d}\phi
		\]
		\end{center}
		& 
		Lorentz-Transformationen werden in der relativistischen Physik benötigt um zwischen verschiedenen, gleichförmig bewegten Koordinatensystem hin und her zu wechseln. Ein einfaches Beispiel ist eine Rotation in der x-y Ebene:\\
		Die Jacobi-Determinante ist die Determinante der Jacobi-Matrix:
		\[
		J= 
		\begin{Vmatrix}
		\frac{\partial x}{\partial r} & \frac{\partial x}{\partial \phi}\\
		\frac{\partial y}{\partial r} & \frac{\partial y}{\partial \phi}
		\end{Vmatrix}
		\]
		& 
		\[
		\begin{pmatrix}
			1 & 0 & 0 & 0 \\
			0 & \cos \vartheta & \sin \vartheta & 0 \\
			0 & -\sin\vartheta & \cos \vartheta & 0 \\
			0 & 0 & 0 & 1
		\end{pmatrix}
		\] \\
		Für eine allgemeine Funktion $f(x_1, \dots, x_n)$ mit $m$ Komponenten sieht die Jacobi-Matrix folgendermaßen aus:
		\[
		\begin{pmatrix}
		\frac{\partial f_1}{\partial x_1} & \cdots & \frac{\partial f_1}{\partial x_n} \\
		\vdots & \ddots & \vdots \\
		\frac{\partial f_m}{\partial x_1} & \cdots & \frac{\partial f_m}{\partial x_n}
		\end{pmatrix}
		\] 
		& oder ein Boost in $x$-Richtung:
		\[
		\begin{pmatrix}
			\cos \vartheta & -\sin \vartheta & 0 & 0 \\
			-\sin \vartheta & \cos \vartheta & 0 & 0 \\
			0 & 0 & 1 & 0 \\
			0 & 0 & 0 & 1 \\
		\end{pmatrix}
		\] \\
		\hline
	\end{tabular}
\end{center}
Andere Konstrukte, die als Matrix gesetzt werden, sind z.\,B. \emph{Wigner-\emph{6j}-Symbole}, z.\,B. in der folgenden Gleichung:
\[
C_3=(-1)^\phi[J,T']^{\frac{1}{2}} 
\begin{Bmatrix}
	k & J' & J \\
	j & T & T'\\
\end{Bmatrix}
\]  
In der relativistischen Quantenmechanik ersetzt man die Quantenzahl $\ell, j$ oft durch die relativistische Drehimpulsquantenzahl (auch: Dirac-Quantenzahl) $\kappa$. Sie ist definiert als 
\[
\kappa=
\begin{cases}
	-(\ell + 1) & \text{für}\ j = \ell + \frac{1}{2} \\
	\ell & \text{für}\ j = \ell - \frac{1}{2}.
\end{cases}
\]
\end{document}