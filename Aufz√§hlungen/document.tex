\documentclass[a4paper. 11pts]{scrartcl}
\usepackage[utf8]{inputenc}
\usepackage [ ngerman ]{ babel }
\usepackage [ T1 ]{ fontenc }
\title{Der Urvater von \TeX \thanks{Ein Großteil dieses Dokuments wurde schamlos aus Wikipedia kopiert.}}
\author{Sebastian Stock}
\date{7. Oktober 2017}



\begin{document}
\maketitle
\noindent \textbf{Donald Ervin Knuth} (* 10. Januar 1938) ist ein US-amerikanischer Informatiker. Er ist Autor des Standardwerkes \textit{The Art of Computer Programming} und Urvater des Textsatzsystems \TeX.\par
Neben Knuths Bemühen um ein ansprechendes ästhetisches Erscheinungsbild beim Textsatz ist ihm Korrektheit ein erstrangiges Anliegen. Deshalb vergibt er für jeden neu gefundenen Fehler in seinen Büchern oder Programmen eine Belohnung von einem \glqq hexadezimalen Dollar \grqq im Wert von \$2,56.

\section{Werke}

Seine beiden  \glqq Hauptwerke\ \grqq \textsc{TAOCP} und \TeX\ (sowie das Begleitbuch \textit{The \TeX book})
sind nicht das einzige, was Knuth in seinem Leben verfasst hat:
\begin{itemize}
	\item Mit seinem Buch \textit{Surreal Numbers: How Two Ex-Students Turned on to Pure Mathematics and Found Total Happines} machte er die von John Horton Conway vorgestellten surrealen Zahlen populär.
	\item Sein Multimediawerk \textit{Fantasia Apocalyptica} (Ein Orgelstück mit Videobegleitung) soll an seinem 80. Geburtstag in Pite\aa, Schweden uraufgeführt werden.
\end{itemize}
\section{The Art of Computer Progamming}
Ursprünglich war das Werk als einzelnes Buch über Compiler geplant. Knuth wollte jedoch alles notwendige Wissen zu diesem Thema präsentieren und dies in einer ausgereiften Form. So entstand der Plan, eine siebenteilige Reihe zu verfassen, die wesentliche Grundlagen der Computerprogramming abdeckt. Die Reihe ist wie folgt geplant:
\begin{enumerate}
	\item Fundamental Algorithms (1968)
	\item Seminumerical Algorithms (1969)
	\item Sorting and Searching
	\item Combinatorial Algorithms (2011)
	\item Syntactical Algorithms (geplant 2025)
	\item The Theory of Context Free Languages
	\item Compilers
\end{enumerate}
Die Beispielprogramme werden in einer von Knuth erdachten Assemblersprache dargestellt, die er für einen fiktiven \glqq idealen \grqq Computer namens \textsc{MIX} entwickelte \footnote{Mit Band 4 wurde \textsc{MIX} durch das "Nachfolgemodell" \textsc{MMIX} abgelöst. Es ist geplant, die Bände 1--3 zu überarbeiten und alle Codebeispiele auf \textsc{MMIX} umzuschreiben.}
Bill Gates hat über das Buch folgendes zu sagen: 
\begin{center}
\emph{``If you think you're a really good programmer\dots\ read (Knuth's) \emph{Art of Computer Programming}\dots\ You should definitely send me a r\'{e}sum\'{e} if you can read the whole thing''}
\end{center}
\end{document}